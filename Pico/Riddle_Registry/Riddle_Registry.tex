\documentclass[12pt, letterpaper]{article}

% Paquete necesario para gestionar fuentes con XeLaTeX
\usepackage{fontspec}

% Configuración de márgenes básicos
\usepackage[margin=1in]{geometry}

% Paquete de inserción de imágenes
\usepackage{graphicx}
\usepackage{tcolorbox}
\usepackage{subcaption}

% Definimos Arial como la fuente principal
\setmainfont{Arial}

\title{Reporte de reto Forense - PicoCTF}
\author{Snaider}
\date{\today}

\begin{document}

\maketitle

\section{Introducción}
Primeramente vemos que tenemos un reto forense el cual se trata de analizar un documento para ver lo que lleva oculto, llevando mas allá nuestros conocimientos en análisis de archivos.
\section{Revisión del documento}
\begin{itemize}

\item Como primera impresión observamos que hay partes censuradas en el documento.

\begin{tcolorbox}[
    colframe=black,
    boxrule=1pt,
    sharp corners,
    boxsep=0pt,
    left=0pt,
    right=0pt
]
\includegraphics[
    trim=2cm 2.9cm 5.3cm 9.2cm,
    clip,
    width=\linewidth,
    height=10cm
]{./img/vista_superficial.png}
\end{tcolorbox}
\par\vspace{1cm}
\item Se puede ver que si seleccionamos la parte censurada y la copiamos a un editor de texto habremos obtenido datos interesantes.
\begin{tcolorbox}[
    colframe=black,
    boxrule=1pt,
    sharp corners,
    boxsep=0pt,
    left=0pt,
    right=10pt
]
    \begin{minipage}{0.5\linewidth}
        \centering
        \includegraphics[width=\linewidth]{./img/textos_censurados.png}
        Textos censurados
    \end{minipage}
    \hfill
    \begin{minipage}{0.47\linewidth}
        \centering
        \includegraphics[width=\linewidth]{./img/textos_descubiertos.png}
        Contenido descubierto de textos censurados
    \end{minipage}

    \vspace{0.5cm}
    \centering   
    \textbf{Descubrimiento de textos censurados}
\end{tcolorbox}
\item Viendo que en el anterior paso no encontramos nada relevante tendremos que intentar otros metodos, intentemos ver los datos en bruto (Raw Data) usando la herramienta \textbf{strings}, haber que encontramos.
    \begin{tcolorbox}[
    colframe=black,
    boxrule=1pt,
    sharp corners,
    boxsep=0pt,
    top=0pt,
    left=0pt,
    right=0pt
]
    \begin{minipage}{\linewidth}
        \centering
        \includegraphics[width=\linewidth]{./img/raw_data.png}
    \end{minipage}

    \vspace{0.2cm}
    \centering   
    \textbf{Datos en bruto}
\end{tcolorbox}

\item Desciframos el cifrado sospechoso
    \begin{tcolorbox}[
    colframe=black,
    boxrule=1pt,
    sharp corners,
    boxsep=0pt,
    top=0pt,
    left=0pt,
    right=0pt
    ]
    \begin{minipage}{\linewidth}
        \centering
        \includegraphics[width=\linewidth]{./img/flag.png}
    \end{minipage}

    \vspace{0.2cm}
    \centering   
    \textbf{Descifrando}

    \end{tcolorbox}

\item Introducimos la flag en el reto de Pico CTF para ver si realmente acabamos el reto.
\begin{tcolorbox}[
    colframe=black,
    boxrule=1pt,
    sharp corners,
    boxsep=0pt,
    top=0pt,
    left=0pt,
    right=0pt
    ]
    \begin{minipage}{\linewidth}
        \centering
        \includegraphics[width=\linewidth]{./img/introduciendo_flag.png}
    \end{minipage}

    \vspace{0.5cm}
    \centering   
    \textbf{Introduciendo flag}

    \end{tcolorbox}

\item Y listo, nos salio las felicitaciones, reto resuelto.

\begin{tcolorbox}[
    colframe=black,
    boxrule=1pt,
    sharp corners,
    boxsep=0pt,
    top=10pt,
    left=10pt,
    right=10pt
    ]
    \begin{minipage}{\linewidth}
        \centering
        \includegraphics[width=\linewidth]{./img/solucionado.png}
    \end{minipage}

    \vspace{0.5cm}
    \centering   
    \textbf{Reto resuelto}

    \end{tcolorbox}
\end{itemize}

\end{document}
