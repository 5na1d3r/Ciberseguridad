\documentclass[12pt, letterpaper]{article}

% --- IDIOMA Y FUENTES ---
% Solución moderna para error de babel en XeLaTeX (Arch Linux)
\usepackage[provide=*]{babel}
\babelprovide[import, main]{spanish}

\usepackage{fontspec}
\setmainfont{Arial}

% --- CONFIGURACIÓN DE TÍTULOS (TODO A 12PT) ---
\usepackage{titlesec}
% Forzamos secciones y subsecciones a 12pt (el segundo número 14 es el interlineado)
\titleformat{\section}{\normalfont\fontsize{12}{14}\bfseries}{\thesection}{1em}{}
\titleformat{\subsection}{\normalfont\fontsize{12}{14}\bfseries}{\thesubsection}{1em}{}

% --- MÁRGENES Y ESTÉTICA ---
\usepackage[margin=1in]{geometry}
\usepackage{graphicx}
\usepackage{tcolorbox}
\usepackage{subcaption}

% Redefinición de \maketitle para que el título también sea 12pt
\makeatletter
\renewcommand{\maketitle}{
  \begin{center}
    {\fontsize{12}{14}\bfseries \@title \par}
    \vspace{0.4cm}
    {\fontsize{12}{14} \@author \par}
    {\fontsize{12}{14} \@date \par}
    \vspace{0.8cm}
  \end{center}
}
\makeatother

% --- DATOS DEL DOCUMENTO ---
\title{Log Hunt - PicoCTF}
\author{Snaider}
\date{\today}

\begin{document}

\maketitle

\section{Introducción}
Primeramente observamos que en el reto ya nos da una pista al decirnos "Our server seems to be leaking pieces of a secret flag in its logs", con eso ya sabemos que tenemos que armar la flag que esta en pedazos dentro del log, demosle un vistazo al archivo.
\section{Revisión del documento}
\begin{itemize}

\item Como primera impresión observamos que hay partes censuradas en el documento.

\begin{tcolorbox}[
    colframe=black,
    boxrule=1pt,
    sharp corners,
    boxsep=0pt,
    left=0pt,
    right=0pt
]
    \centering
    \vspace{0.2cm}
    \textbf{Textos censurados}
\end{tcolorbox}

\par\vspace{1cm}

\item Se puede ver que si seleccionamos la parte censurada y la copiamos a un editor de texto habremos obtenido datos interesantes.

\begin{tcolorbox}[
    colframe=black,
    boxrule=1pt,
    sharp corners,
    boxsep=0pt,
    left=0pt,
    right=10pt
]
    \begin{minipage}{0.5\linewidth}
        \centering
        \small Textos censurados
    \end{minipage}
    \hfill
    \begin{minipage}{0.47\linewidth}
        \centering
        \small Contenido descubierto
    \end{minipage}

    \vspace{0.5cm}
    \centering   
    \textbf{Descubrimiento de textos censurados}
\end{tcolorbox}

\item Viendo que en el anterior paso no encontramos nada relevante tendremos que intentar otros métodos, intentemos ver los datos en bruto (Raw Data) usando la herramienta \textbf{strings}, a ver qué encontramos.

\begin{tcolorbox}[
    colframe=black,
    boxrule=1pt,
    sharp corners,
    boxsep=0pt,
    top=0pt,
    bottom=7pt,
    left=0pt,
    right=0pt
]
    \begin{minipage}{\linewidth}
        \centering
    \end{minipage}
    \centering  
        \vspace{0cm}

    \textbf{Datos en bruto}
\end{tcolorbox}

\item Desciframos el cifrado sospechoso.
\begin{tcolorbox}[
    colframe=black,
    boxrule=1pt,
    sharp corners,
    boxsep=0pt,
    top=0pt,
    left=0pt,
    right=0pt
]
    \begin{minipage}{\linewidth}
        \centering
    \end{minipage}
    
    \centering
    \vspace{0.2cm}
    \textbf{Descifrando}
\end{tcolorbox}

\item Introducimos la flag en el reto de Pico CTF para ver si realmente acabamos el reto.
\begin{tcolorbox}[
    colframe=black,
    boxrule=1pt,
    sharp corners,
    boxsep=0pt,
    top=0pt,
    left=0pt,
    right=0pt
]
    \begin{minipage}{\linewidth}
        \centering
    \end{minipage}
    \vspace{0.5cm}
    \centering   
    \textbf{Introduciendo flag}
\end{tcolorbox}

\item Y listo, nos salió las felicitaciones, reto resuelto.

\begin{tcolorbox}[
    colframe=black,
    boxrule=1pt,
    sharp corners,
    boxsep=0pt,
    top=10pt,
    bottom=10pt,
    left=10pt,
    right=10pt
]
    \begin{minipage}{\linewidth}
        \centering
    \end{minipage}
    \centering 
        \vspace{0.1cm}

    \textbf{Reto resuelto}
\end{tcolorbox}

\end{itemize}

\end{document}
